\documentclass[11pt,ngerman]{report}

\usepackage[utf8]{inputenc}
\usepackage[T1]{fontenc}
\usepackage[ngerman]{babel}
\usepackage[at]{easylist}
\usepackage{graphicx}
\usepackage{tabulary}
\usepackage{multirow}
\usepackage{booktabs}
\usepackage{hyperref}

\setlength\parindent{0pt}
\renewcommand{\baselinestretch}{1.5}

% Title Page
\title{Konzipierung und Implementierung eines geschwindigkeitsbasierten Reaktionsspiels unter Verwendung des 8051 Mikrocontrollers}
\author{Grundmann, Alexander\\Köhler, Sven\\Schneble,Adrian\\Weickenmeier, Marc}


\begin{document}
	
%%%%%%%%%%%%%%%%%%%%%%%%%%%%%%%%%%%%%%%%%%%%
\maketitle

%%%%%%%%%%%%%%%%%%%%%%%%%%%%%%%%%%%%%%%%%%%%
\tableofcontents


%%%%%%%%%%%%%%%%%%%%%%%%%%%%%%%%%%%%%%%%%%%%
\chapter{Einleitung}

\section{Motivation}

Ziel dieses Projekts ist vorrangig die Beschäftigung mit dem 8051 Mikrocontroller.
In diesem Kontext wurde von uns demokratisch die Entwicklung eines geschwindigkeitsbasierten Reaktionsspiels beschlossen.

\section{Aufgabenstellung}

Ziel dieses Projekts ist die Erstellung eines Programms für den 8051 Mikrocontroller unter Verwendung der MCU 8051 IDE, welche die Hardwarekomponenten simuliert.
Dies ermöglicht es, die Software ohne einen physisch vorhandenen Mikrocontroller schnell und wiederholt auszuführen.



%%%%%%%%%%%%%%%%%%%%%%%%%%%%%%%%%%%%%%%%%%%%
\chapter{Grundlagen}

\section{Assembler}



\section{Der 8051 Mikrocomputer}



\section{Entwicklungsumgebung MCU-8051 IDE}



%%%%%%%%%%%%%%%%%%%%%%%%%%%%%%%%%%%%%%%%%%%%
\chapter{Konzept}

\section{Analyse}



\section{Programmentwurf}



%%%%%%%%%%%%%%%%%%%%%%%%%%%%%%%%%%%%%%%%%%%%
\chapter{Implementierung}



%%%%%%%%%%%%%%%%%%%%%%%%%%%%%%%%%%%%%%%%%%%%
\chapter{Fazit}




\end{document}          
